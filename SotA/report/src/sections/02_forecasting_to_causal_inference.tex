\section{From Forecasting to Causal Inference}
% \section{Limits of Forecasting}
\label{sec:forecasting_to_causal_inference}

\subsection{The State of the Art in Conflict Forecasting}
Contemporary conflict forecasting has moved beyond small-scale country-specific desings used in earlier research \cite{cederman_predicting_2017, hegre_introduction_2017}.
Trough efforts, such as the Violence and Impacts Early-Warning System (VIEWS) \cite{hegre_views_2019} or ACLED Volatility Risk Index/CAST \cite{rod_review_2024} and a "revolution" in data as well as methodology, the scope and predictive performance of forecasting in conflict research increased substantially \cite{croicu_forecasting_2025}.
Current data---led by datasets such as the Uppsala Conflict Data Program's Georeferenced Event Dataset (UCDP GED) \cite{sundberg_introducing_2013} or the Armed Conflict Event and Location Dataset (ACLED) \cite{raleigh_introducing_2010}---caputers local dynamics troguh global collections of individual armed conflict events on a daily basis over a long time-series.
This highpy disaggregated microdata with millions of data-points is then often leveraged to perform a spatio-temporal aggregation; typically a PRIO-GRID lattice \cite{tollefsen_prio-grid_2012} with $0.5^\circ \times 0.5^\circ$ spatil grid cells ($55 \times 55$ km$^2$ at the equator) and a temporal resolution of a month is used \cite{hegre_202324_2025, hegre_views_2019}.
These event-based frameworks are usually complemented with structural covarites---static, long-term data like economic and social indicators---to combine conflict dynamics with underlying socio-economic conditions \cite{tollefsen_prio-grid_2012, hegre_views_2019, rod_review_2024}.
Furthemore, integration of remote sensing data variables (e.g., vegetation indices, satillite imagery) have further enhanced this spatial granularity \cite{mueller_monitoring_2021, mhanna_using_2023, racek_conflict_2024}.

Recent models perform well at static risk assessment but have limited capabilities at predicting conflict dynamics (e.g., onsets, escalations, and terminations) \cite{mueller_monitoring_2021, vesco_united_2022}.
Methodologically, most of today's approaches are using non-parametric ensembles(e.g., XGBoost or Random Forests) that excel at capturing non-linear interactions and minimizing predictive error (e.g., Brier scores) \cite{hegre_views_2019, hegre_views2020_2021}.
A current evolution in the field is moving from manual feature engineering to deep learning, with some frontier models adopting architectures that autonomously learn representations and capture intricate long-term temporal dependencies \cite{croicu_forecasting_2025}.
Examples include Long Short-Term Memory (LSTM) networks, Recurrent Neural Networks (RNNs), and Transformers, such as Temporal Fusion Transformers (TFTs) \cite{hegre_202324_2025, vesco_united_2022}.
Another emerging approach is direct forecasting from news corpora by using Transformer-based Large Language Models (LLMs) and Retrieval-Augmented Generation (RAG) \cite{croicu_newswire_2025, nemkova_large_2025}.
While such "black box" ML models maximize metrics like AUROC, they lack interpretability and obscure structural mechanisms \cite{murphy_promise_2024, cederman_predicting_2017}.
Most empirical studies also treat conflict diffusion simply as a nuisance, controlling for it by including simple spatial and temporal lags; this leads to very limited insights on the diffusion of armed conflict \cite{racek_capturing_2025}.
Research by \textcite{racek_capturing_2025} addresses these gaps by utilizing a Generalized Additive Model (GAM) with non-parametric smoothing.
Therefore, the model can capture the spatio-temporal diffusion of conflict over large distances and temporal lags of up to 24 months while remaining fully interpretable.

\subsection{The Missing Causal Underestanding}
The majority of current research in the field is focused on forecasting, with models designed to maximize predictive accuracy over interpretability, leaving the understanding of conflict's root causes and treatment evaluation underexplored \cite{murphy_promise_2024, cederman_predicting_2017}.
Yet, as \textcite[23]{cederman_predicting_2017} argue, "theory-free prediction does little to guide intervention without knowledge about the drivers of conflict". %???
Since prediction and explanation represent distinct epistemological goals, a model that excels at forecasting with high accuracy may reveal nothing about underlying mechanisms and relations \cite{shmueli_explain_2010}.
Therefore, architecture selection often involves a trade-off between model accuracy and explanatory capabilities \cite{breiman_statistical_2001}.
Even if the prediction accuracy of a model is high---for instance, successfully forecasting increasing levels of violence---policymakers may lack trust to take action and authorize costly interventions \cite{chadefaux_conflict_2017, rudin_stop_2019}.
Furthermore, even fully interpretable models like GAMs cannot differentiate between causation and correlation; as \textcite{cederman_predicting_2017} note, forecasting models identify correlates of conflict, not causes.
Crucially, policy decisions require counterfactual reasoning, which purely observational forecasting models cannot answer \cite{feuerriegel_causal_2024}.
Relying on associations without considering confounding or reverse causality can therefore lead to misguided policies and interventions (e.g., erroneous foreign aid allocation, ineffective military actions) \cite{nunn_us_2014, papadogeorgou_causal_2022}.
Causal inference, specifically Causal Machine Learning (CML), bridges this gap by enabling valid statistical inference for treatment effects while utilizing high-dimensional data \cite{athey_machine_2019, feuerriegel_causal_2024}.

\subsection{Causal Machine Learning}
Causal effect estimation is challenging because of the inability to observe both treated and untreated potential outcomes for the same unit simultaneously; this is generally referred to as the fundamental problem of causal inference \cite{holland_statistics_1986}.
CML addresses this by combining the flexible estimation power of ML with principles from causal inference \cite{athey_machine_2019, chen_recent_2025}.
While traditional econometric approaches struggle with high-dimensional covariate spaces \cite{belloni_high-dimensional_2014} and standard ML methods introduce regularization bias \cite{feuerriegel_causal_2024, kaddour_causal_2022}, CML builds on the potential outcomes framework \cite{rubin_estimating_1974} to formalize causal inference trough counterfactual reasoning and relax restrictive parametric assumptions of traditional regressions \cite{athey_machine_2019, fuhr_estimating_2024}.
This means explicitly modeling the confounding structure by flexibly approximating nuisance functions and adjusting for high-dimensional confounding without restrictive parametric assumptions \cite{feuerriegel_causal_2024}.
The framework therefore allows for valid effect estimation in environments with a high number of covariates \cite{belloni_high-dimensional_2014} and complex non-linear relations \cite{fuhr_estimating_2024}.
