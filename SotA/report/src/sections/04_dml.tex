\section{Double Machine Learning}
\label{sec:dml}

\subsection{The Basics}
\label{subsec:basics}
The high-dimensional setting of conflict data, where the number of potential confounders is large relative to the number of observations, necessitates the use of ML methods.
While standard ML methods achieve strong predictive performance through regularization, the latter also introduces systematic bias in treatment effect estimates that does not vanish with increasing sample size.
Double Machine Learning (DML)---formally introduced by \textcite{chernozhukov_doubledebiased_2018} in 2018---resolves the regularization bias problem and allows for valid causal inference combined with flexible ML methods.

To illustrate the concept of DML, consider the following Partially Linear Regression (PLR) model \cite{robinson_root-n-consistent_1988}:
\begin{alignat*}{2}
Y &= D\theta_0 + g_0(X) + U, \quad &\hspace{\fill} \mathbb{E}[U|X,D] &= 0 \\
D &= m_0(X) + V, &\hspace{\fill} \mathbb{E}[V|X] &= 0
\end{alignat*}
where $Y$ denotes the outcome (e.g., conflict fatalities), $D$ the treatment (e.g., military intervention, aid allocation), $U$ and $V$ disturbances, and $X \in \mathbb{R}^p$ a high-dimensional vector of confounders (e.g., GDP, population density, conflict event data).
$\theta_0$ is the causal parameter of interest that captures the causal effect of $D$ on $Y$, while the nuisance functions $g_0(X)$ and $m_0(X) = \mathbb{E}[D|X]$ (the propensity score) are unknown and potentially complex.
The challenge lies in making a valid estimation of $\theta_0$ despite the high dimensionality of the nuisance functions $\eta_0 = (g_0, m_0)$.

Simply using a flexible ML algorithm like a Random Forest to estimate $\theta_0$ by (1) estimating $g_0$ and (2) regressing the residuals $Y - \hat{g}(X)$ on $D$ leads to biased results.
Modern ML methods rely on regularization to manage high-dimensional data, which introduces regularization bias;
the estimation error $\hat{g}(X) - g_0(X)$ remains correlated with $D$ through their mutual dependence on $X$.
This bias propagates directly into $\hat{\theta}$, does not vanish at rate $1/\sqrt{n}$, and invalidates standard inference.

DML resolves this by employing Neyman orthogonality and constructing an orthogonal score that is insensitive to first-order errors in nuisance estimation.
In addition to partialling out $X$ from $Y$, DML also residualizes the treatment.
This isolates the variation in $D$ and $Y$ that is not explained by $X$ by estimating the nuisance functions $\hat{g}(X)$ and $\hat{m}(X)$ to create residuals $\tilde{Y} = Y - \hat{g}(X)$ and $\tilde{D} = D - \hat{m}(X)$.
Formally, this estimator depends on a score function $\psi(W; \theta, \eta)$ that is insensitive to small perturbations in the nuisance parameters $\eta$.
In the PLR example, the orthogonal score function takes the form:
\begin{equation*}
    \psi(W; \theta, \eta) = (Y - g(X) - D\theta)(D - m(X))
\end{equation*}
where $W = (Y, D, X)$ and $\eta = (m, g)$.
This satisfies the Neyman orthogonality condition:
\begin{equation*}
\partial_\eta \mathbb{E}[\psi(W; \theta_0, \eta_0)][\eta - \eta_0] = 0
\end{equation*}
This condition ensures that first-order errors in estimating $\hat{g}$ and $\hat{m}$ vanish, leaving only second-order errors that do not affect the asymptotic distribution of $\hat{\theta}$.

To prevent overfitting, DML utilizes cross-fitting: (1) the sample is randomly partitioned into $K$ folds;
(2) the nuisance functions are estimated on the remaining $K-1$ folds;
(3) the orthogonal score is evaluated on the held-out fold;
(4) the results are either averaged across folds (DML1) or pooled into a single estimating equation solved over all observations (DML2).
Under regularity conditions, this separation guarantees that the final DML estimator is $\sqrt{n}$-consistent and asymptotically normal:
\begin{equation*}
    \sqrt{n}(\hat{\theta} - \theta_0) \xrightarrow{d} \mathcal{N}(0, \sigma^2)
\end{equation*}
provided that the nuisance estimators converge at a rate of $n^{-1/4}$ or faster.

\subsection{DML for Dynamic Treatment Effects}
\label{subsec:dml_dte}
The standard DML framework as described in \Cref{subsec:basics} addresses regularization bias in static settings.
However, applying it to spatio-temporal conflict data introduces new challenges---which were discussed in \Cref{sec:spatio_temporal}---that require specialized extensions.
\textcite{lewis_doubledebiased_2021} (see also \cite{lewis_doubledebiased_2021-1}) were among the first researchers to introduce a solution for DML with Dynamic Treatment Effects (DTE), which focuses on dealing with temporal persistence and feedback loops (see \Cref{subsec:spatial_dynamic_confounding});
their proposed Dynamic DML estimation extends the Neyman orthogonality principle to dynamic treatment schemes (settings where multiple treatments are assigned over time and where treatments can have a causal effect on future outcomes or the state of the treated unit).

A partially linear state space Markov decision process $\{X_t, D_t, Y_t\}_{t=1}^{m}$ (for a time horizon $t=1, \dots, m$) where $X_t \in \mathbb{R}^p$ denotes the high-dimensional state, $D_t \in \mathbb{R}^d$ the treatment, and $Y_t$ the outcome (at time $t$ respectively) will serve as a simplified application example of the algorithm, similar to the PLR used in \Cref{subsec:basics}.
The structural equations take the form:
\begin{align*}
    X_t &= A \cdot D_{t-1} + B \cdot X_{t-1} + W_t \\
    D_t &= m(D_{t-1}, X_t) + V_t \\
    Y_t &= \theta_0' D_t + \mu' X_t + U_t
\end{align*}
where $W_t, V_t, U_t$ are disturbances, $A$ captures how treatments affect future states, $B$ governs how past states affect the next period’s state, and $\theta_0$ is the contemporaneous treatment effect.
Further, $\mu$ is the contemporaneous effect of the states on the outcome and $m$ is the treatment assignment function (propensity).
The goal is to estimate the effect of a change in the treatment policy on the final outcome $Y_m$.
Therefore, the sequence of dynamic treatment effects $\theta_t$ for $t\in\{1, \dots, m-1\}$ with $\theta_t = \mu' B^{t-1} A$ ($\theta_t$ represents the causal effect of a treatment impulse at time $m-t$ on the outcome at time $m$) can be used to capture effects operating through state changes.
To estimate the expected difference between applying and not applying the treatment, it suffices to estimate the full sequence of dynamic treatment effects $\theta_0, \dots, \theta_m$.

To address the estimation problem, Dynamic DML builds on the principles from \Cref{subsec:basics} by constructing Neyman orthogonal moments at each stage via double residualization.
The key innovation is a sequential peeling strategy that estimates effects backwards from the final period $m$ to the first period;
the algorithm proceeds for $t=0,1,\dots,m$---and assuming access to a dataset with $n$ i.i.d. samples from the Markovian process---as follows:
\begin{enumerate}
    \item \textbf{Nuisance Estimation:} For the current lag $t$, estimate the conditional expectations of the outcome and all relevant treatments with respect to the current history $X_{m-t}$.
    Recalculating the treatment residuals is necessary because treatment residuals from previous steps were calculated conditional on future states.
    Fit the nuisance model for the outcome $\hat{g}_t(x) = \mathbb{E}[Y_m \mid X_{m-t}=x]$ and for each $j\in\{0,\dots,t\}$ the treatment model $\hat{m}_{j,t}(x) = \mathbb{E}[D_{m-j} \mid X_{m-t}=x]$ using cross-fitting.
    Finally, compute the corresponding residuals $\tilde{Y}_{m,m-t}$ and $\tilde{D}_{m-j, m-t}$.

    \item \textbf{Peeling:} Remove the causal contribution of all subsequent treatments using previously estimated effects $\hat{\theta}_0, \ldots, \hat{\theta}_{t-1}$ to isolate the variation in the outcome residuals that is attributable to the treatment at $m-t$.
    Thus, the ``calibrated'' outcome residual $\bar{Y}_{m,t}$ is calculated by subtracting the weighted future treatment residuals:
    \begin{equation*}
        \bar{Y}_{m,t} = \tilde{Y}_{m,m-t} - \sum_{j=0}^{t-1} \hat{\theta}_j' \tilde{D}_{m-j, m-t}
    \end{equation*}

    \item \textbf{Orthogonal Estimation:} Estimate $\theta_t$ by solving the Neyman orthogonal moment condition:
    \begin{equation*}
        \frac{1}{n} \sum_{i=1}^{n} (\bar{Y}_{m,t}^i - \theta_t \tilde{D}_{m-t, m-t}^i) \tilde{D}_{m-t, m-t}^i = 0
    \end{equation*}
\end{enumerate}

% The algorithm, as shown above, assumes the availability of $n$ independent time series of duration $m$ from the Markovian process.
% In conflict research, this assumption is rarely met since the data often exhibits strong temporal autocorrelation as mentioned in \Cref{subsec:statistical_dependence}.
% \textcite{lewis_doubledebiased_2021} propose a solution to this by estimating the DTEs for a fixed look-ahead horizon from a single long time-series.
% They split the time-series into sequential blocks, which are treated as roughly independent observations.
% Afterwards, a progressive nuisance estimation fitting approach is employed, where at every period, all prior blocks are used to train the nuisance models, which are then evaluated on the next block.
% Since the effect of current interventions on future outcomes vanishes over time due to discounting, the required look-ahead horizon $m$ only needs to grow logarithmically with the sample size ($m \approx \log_{1/\gamma} n$ with $\gamma < 1$).
% This logarithmic growth suffices to bound the approximation error to a negligible order of $1/\sqrt{n}$, ensuring the block-based approach remains valid asymptotically.
% The only caveat is that for correct inference, the underlying systems must be somewhat stationary and stable, ensuring that the dependencies between blocks vanish sufficiently fast to prevent estimation errors from propagating exponentially.

Furthermore, the framework generalizes beyond the Markovian case utilized thus far to Structural Nested Mean Models (SNMMs) \cite{robins_new_1986} with minimal modifications.
In this setting, restrictive independence assumptions are replaced by the weaker condition of sequential conditional exogeneity;
this essentially requires that treatments are randomized conditional on past history.
SNMMs also allow for arbitrary state spaces and discrete or continuous treatments, employing blip functions for identification.
These functions capture the expected difference in the outcome if the treatment is removed, while continuing with a target policy thereafter.
Nevertheless, applying the framework to empirical settings like conflict research remains challenging due to multiple factors.
For instance, even after easing the assumption to sequential conditional exogeneity, conflict settings might still violate it since unobserved strategic factors that drive both treatment and outcome (spatial confounding) are often present.
The model also does not take spatial spillovers into account, thereby violating SUTVA, essential for valid causal inference.

\subsection{DML for Heterogeneous Treatment Effects in Dynamic Panels}
\label{subsec:dml_hte}
The frameworks presented in \cref{subsec:basics} and \cref{subsec:dml_dte} estimate the ATE\footnote{For both, HTE estimation exists.
See \cite{semenova_debiased_2021} for the DML CATE extension and the original DML for DTE paper \cite{lewis_doubledebiased_2021}.}.
For policy applications or to analyze conflict diffusion, understanding how treatment effects vary throughout contexts and characteristics is important.
While the dynamic DML framework (\Cref{subsec:dml_dte}) addresses temporal feedback loops, conflict data presents additional challenges (see \cref{sec:spatio_temporal}).
\textcite{semenova_inference_2023} (also \cite{semenova_estimation_2022}) present an inference approach that tackles both of these aspects.
Their method estimates the CATE---which captures treatment effect heterogeneity as a function of unit characteristics---and works with high-dimensional panels;
they specifically address static spatial confounding (unobserved unit-specific heterogeneity/unit fixed effects) and statistical dependence (weak dependence).

Consider the following adjustment to the PLR model from \cref{subsec:basics} that allows for heterogeneous effects ($i = 1,\dots,N$ denotes units, and $t=1,\dots,T$ time):
\begin{alignat*}{2}
    Y_{it} &= D_{it}' \theta_0 + e_0(X_{it}) + \xi_i^E + U_{it}, \quad &\hspace{\fill} \mathbb{E}[U_{it} | D_{it}, X_{it}, \Phi_{it}] &= 0 \\
    D_{it} &= m_{i0}(X_{it}) + V_{it}, &\hspace{\fill} \mathbb{E}[V_{it}| X_{it},\Phi_{it}] &= 0
\end{alignat*}
where $\theta_0$ is the treatment effect parameter vector characterizing the CATE, $\Phi_{it}$ is the filtration of predetermined variables for unit $i$ prior to period $t$, $D_{it}$ is the technical treatment vector (defined as the interaction between the base treatment $P_{it}$ (the vector with policy variables) and the dictionary of transformations $K(X_{it})$: $D_{it} := P_{it} \cdot K(X_{it})$), $e_0(X_{it})$ is the complex confounding function of time-varying controls, and $\xi_i^E$ is the unobserved outcome unit fixed effect (fixed spatial confounding).
The error term $U_{it}$ is assumed to satisfy sequential conditional exogeneity.

Standard fixed effects estimation is problematic in high-dimensional dynamic settings since it often leads to biased results and overfitting.
The authors therefore take a fixed effects approach in which they approximate the vector of unobserved components of unit effects $\xi^E$ by a weakly sparse vector.
Instead of treating the effects as a nuisance to be differenced out, the method estimates them using Lasso.
This assumes that the unit-specific deviations are well-approximated by a sparse vector, allowing them to be learned consistently alongside other nuisance parameters even when the number of units is large.

For the estimation of the high-dimensional parameter $\theta_0$, the DML principle of orthogonalization is applied to partial out the influence of confounders and unit effects.
The unit-specific nuisance functions are defined as:
\begin{align*}
    m_{0i}(X_{it}) &= \mathbb{E}[D_{it} | X_{it}, \Phi_{it}] = d_0(X_{it}; \xi_i), \\
    g_{i0}(X_{it}) &= \mathbb{E}[Y_{it} | X_{it}, \Phi_{it}] = m_{i0}(X_{it})'\theta_0 + e_0(X_{it}) + \xi_i^E
\end{align*}
with $\xi$ denoting a fixed vector of unit-specific fixed treatment selection effects.
The residuals $V_{it} := D_{it} - d_{i0}(X_{it})$ and $\tilde{Y}_{it} := Y_{it} - g_{i0}(X_{it})$ then result in the orthogonalized equation:
\begin{equation*}
    \tilde{Y}_{it} = V_{it}' \theta_0 + U_{it}, \quad \mathbb{E}[U_{it} | V_{it}, X_{it}, \Phi_{it}] = 0,
\end{equation*}
which identifies $\theta_0$ as the coefficient of the best linear projection of $\tilde{Y}_{it}$ on $V_{it}$.

As already mentioned in \cref{subsec:statistical_dependence}, regular cross-fitting does not work for the assumed data structure.
\textcite{semenova_inference_2023} resolve this with Neighbors-Left-Out (NLO) cross-fitting, where the time series is partitioned into $K$ adjacent blocks $\{M_k\}_{k=1}^K$.
Then temporal neighbors are excluded by defining a new quasi-complement for each block as $M_k^{qc} = \{M_1, \ldots, M_K\} \setminus \{M_l : l \in \mathcal{N}(k)\}$, where $\mathcal{N}(k)$ denotes $k$ and its immediate neighbors.
Consequently, the method ensures approximate independence between training and test samples.

The complete algorithm consists of three steps:
\begin{enumerate}
\item \textbf{Nuisance Estimation:} Estimate the nuisance functions $\hat{g}$ and $\hat{m}$ using appropriate modeling structures and ML methods that can handle fixed effects (for instance, Lasso for the sparse fixed effects) as well as NLO cross-fitting.
Afterwards, calculate the residuals $\widehat{\tilde{Y}}_{it}$ and $\widehat{V}_{it}$.
\item \textbf{Orthogonal Estimation:} Estimate the causal effect parameter vector (the CATE function) $\hat{\theta}_0$ by regressing $\tilde{Y}_{it}$ on $V_{it}$ using Lasso:
    \begin{equation*}
        \hat{\theta} = \arg\min_{\theta \in \mathbb{R}^d} \frac{1}{NT} \sum_{i,t} \left( \widehat{\tilde{Y}}_{it} - \widehat{V}_{it}' \theta \right)^2 + \lambda_\theta \|\theta\|_1.
    \end{equation*}
where $\lambda_\theta=C_\theta\sqrt{\log d/NT}$ and $C_\theta$ is a penalty parameter.
This selects the relevant interaction terms that drive treatment heterogeneity.
\item \textbf{Debiased Inference:} To correct for the regularization bias introduced in step 2, construct valid Gaussian confidence intervals for CATE using debiased Lasso with an approximate inverse of the residual covariance matrix.
\end{enumerate}

Similar to the dynamic DML framework in \cref{subsec:dml_dte}, applying this approach to conflict data comes with potential challenges.
For instance, the weak sparsity assumption on residual unit effects $\xi^E$ may be violated by real-world data when every spatial unit has unique dense confounders.
While the model addresses temporal autocorrelation, it assumes cross-sectional independence between units.
The framework also does not explicitly address spatial spillovers, which is essential for conflict research (\cref{subsec:sutva}).

\subsection{Other Literature}
Other recent literature has further expanded DML for dynamic and panel data settings.
In environments where unobserved time-invariant unit-specific confounding is the primary concern, \textcite{clarke_double_2025} extend DML to static panels with fixed effects.
They demonstrate an adaptation of the Neyman orthogonal score to remove unobserved heterogeneity while allowing for complex nuisance functions.
\textcite{bodory_evaluating_2022} introduce a weighting-based DML extension strategy for estimating dynamic treatment effects that robustly adjusts for time-varying confounding across treatment sequences.
Another recent paper that is directly relevant for the context of conflict diffusion develops DML estimators for impulse response functions in local projections, which model how treatment effects propagate over multiple time periods \cite{ballinari_semiparametric_2025}.
